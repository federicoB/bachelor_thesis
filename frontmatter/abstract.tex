% Description: Contains the text for thesis abstract
%
% Abstract guidelines:
%
% Each copy must include an abstract or summary of the dissertation in not
% more than 300 words, on one side of A4, which should be single-spaced in a
% font size in the range 10 to 12. If the dissertation is in a language other
% than English, an abstract in that language and an abstract in English must
% be included.

\chapter*{Abstract}
\begin{SingleSpace}

The increasingly massive collection of data from various types of sensors installed on vehicles allows the study and reconstruction of their dynamics in real time, as well as their archiving for subsequent analysis.
This Thesis describes the design of a numerical algorithm and its implementation in a program that uses data from inertial and geo-positioning sensors, with applications in industrial contexts and automotive research. The result was made usable through the development of a Python add-on for the Blender graphics program, able to generate a three-dimensional view of the dynamics that can be used by experts and others.
Throughout the Thesis, particular attention was paid to the complex nature of the data processed, introducing adequate systems for filtering, interpolation, integration and analysis, aimed at reducing errors and simultaneously optimizing performances.

\vspace{20mm}

La raccolta sempre più massiccia di dati provenienti da sensori di varia natura installati sui veicoli in circolazione permette lo studio e la ricostruzione della loro dinamica in tempo reale, nonché la loro archiviazione per analisi a posteriori. 
In questa Tesi si descrive la progettazione di un algoritmo numerico e la sua implementazione in un programma che utilizza dati provenienti da sensori inerziali e di geo-posizionamento con applicazioni a contesti industriali e di ricerca automobilistica. Il risultato è stato reso fruibile tramite lo sviluppo di un add-on Python per il programma di grafica Blender, in grado di generare una visualizzazione tridimensionale della dinamica fruibile da esperti e non. 
Durante tutto il lavoro di Tesi, particolare attenzione è stata prestata alla complessa natura dei dati trattati, introducendo adeguati sistemi di filtraggio, interpolazione, integrazione ed analisi, volti alla riduzione degli errori e alla contemporanea ottimizzazione delle prestazioni.

\end{SingleSpace}