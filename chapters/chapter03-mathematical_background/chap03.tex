%\let\textcircled=\pgftextcircled
\chapter{Mathematical background}
\label{chap:math_background}

\initial{I}n this chapter I will introduce some mathematical and physics notion used in the project.
The first part of the chapter deals with numerical integration, a tecnique used to computationally calculate the integral of physics quantities like acceleration and velocity to get the trajectory traveled by the vehicle.
The second part deals with quaternion, a number system useful to handle rotations in tridimensional space more efficient and numerically stable than rotation matrices.

\section{Numerical integration}

\section{Quaternions}
Quaternions are an extension of complex numbers usually represented in the form:
$$ a + xi + yj + wk $$
where $a, x, y, w$ are real numbers and $i, j, k$ are called \textit{quaternion units} \\
with the property ${ijk=-1}$  \\
$a$ is also called \textit{scalar} or \textit{real} part of the quaternion, while $x,y,w$ is the tridimensional \textit{vector} or \textit{imaginary} part. \cite{amslaurea6701} \\
Quaternions have been formalized by the Irish mathematician William Hamilton in 1843. \\
If the scalar part is equal zero then the quaternion is called \textit{pure}.

\subsection{Vector rotations with quaternions}
Given a point $\boldsymbol{v} \in \mathbb{R}^3$ its rotation around versor $\boldsymbol{x}=\{a_x, a_y, a_z\}$ by an angle $\theta$ can be obtained in the following way:
$$ p' = q p \overline{q}$$
where: \\
\begin{itemize}
\item $p$ is a pure quaternion with the vector part equal to $\boldsymbol{v}$
\item $q$ is a quaternion representing rotation $q=e^{\frac{\theta}{2}(a_x i + a_y j + a_z k)}=\cos(\frac{\theta}{2})+(a_x i + a_y j + a_z k)\sin(\frac{\theta}{2})$ 
\item $\overline{q}$ is the conjugation of quaternion $q$ definied as \\
$\overline{q}=(s,-\boldsymbol{v})=s-xi-yj-wk=e^{-\frac{\theta}{2}(a_x i + a_y j + a_z k)}=\cos(\frac{\theta}{2})-(a_x i + a_y j + a_z k)\sin(\frac{\theta}{2})$
\item $p'$ is a pure quaternion with its vector part equal to vector $\boldsymbol{v}$ rotated
\end{itemize}

\subsection{Product of quaternions}
Given two quaternion $q_1$ and $q_2$ defined as: \\
$q_1 = (s_1,\boldsymbol{v_1})=s_1+a_1i+b_1j+c_1k$ \\
$q_2=(s_2,\boldsymbol{v_2})=s_2+a_2i+b_2j+c_2k$ \\
their product can be defined as: \\
$q_1 q_2 = (s_1 s_2 - \boldsymbol{v_1 \cdot v_2}, \ s_1 \boldsymbol{v_2} + s_2 \boldsymbol{v_1} + \boldsymbol{v_1} \times \boldsymbol{v_2})$

\subsection{Quaternion product isn't commutative}
$q_1 q_2 \neq q_2 q_1$  \\
is equivalent to \\
$(s_1 s_2 - \boldsymbol{v_1 \cdot v_2}, \ s_1 \boldsymbol{v_2} + s_2 \boldsymbol{v_1} + \boldsymbol{v_1} \times \boldsymbol{v_2}) \neq $
$(s_1 s_2 - \boldsymbol{v_1 \cdot v_2}, \ s_1 \boldsymbol{v_2} + s_2 \boldsymbol{v_1} + - (\boldsymbol{v_1} \times \boldsymbol{v_2}))$ \\
by the anticommutative property of vector product \cite{amslaurea6701} \\
$$ \boldsymbol{a} \times \boldsymbol{b} = - (\boldsymbol{b} \times \boldsymbol{a}) $$ 

\subsection{Comparison with rotation matrices}
Matrices can also be used to perform rotation. \cite{} given a vector $\boldsymbol{v}$, a rotation angle $\theta$ and some versor $\boldsymbol{x}=\{a_x, a_y, a_z\}$  \\
\textit{Optional TODO: show how to get a matrix from axis angle and how to make rotations} \\
A quaternion can be associated with a rotarion matrix and and vice versa. \cite{amslaurea6701,Eberly2016RotationRA} \\
Aside from memory usage, where quaternions are more efficient, the time complexity must take account conversion from rotation system, as from axis-angle in our case, and actual context of usage as explained after.  \\
Performance can be evaluated at high-level by counting the numer of operation needed to perform conversion and rotations. Operations like additions or subtractions, multiplications, divisions and trigonometric function evaluations like sines and cosines. \\
In the following analysis i will leave out performance optimization that computer systems can apply in some situations. Just looking at number of operations should give a general idea of difference between rotation representations. \\
Axis-angle to matrix requires 13 additions, 15 multiplications and 2 trigonometric calls. \\
Axis-angle to quaternion requires 4 multiplications and 2 trigonometric calls. \\
Rotating a vector using a rotation matrix requires 5 additions and 9 multiplications. \\
Rotating a vector using a quaternion, optimizing using the fact that scalar part is zero, requires 17 additions and and 24 multiplications. \\
So it results that rotation matrix are more efficient in rotating a vector. \\
The real advantage of quaternion are rotation composition. 
Indeed, composing two quaternion require only 12 additions and 16 multiplications while matrices 18 additions and 27 multiplications. \\
In the end, having n vectors to rotate by a cumulative n angles represented as axis-angle notation the following comparison can be made:
\begin{itemize}
\item rotation matrices: ${((13A + 15M + 2F) + (5A + 9M) + (18A + 27M))n} = {(36A + 51M + 2F)n}$
\item quaternions: ${((4M + 2F) + (17A + 24M) + (12A + 16M))n} = {(29A + 44M + 2F)n}$
\end{itemize}
% faster converting to matrix than using quaternions

