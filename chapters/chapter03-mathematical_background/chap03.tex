\let\textcircled=\pgftextcircled
\chapter{Mathematical background}
\label{chap:math_background}

\initial{I}n this chapter I will introduce some mathematical and physics notion used in the project.
The first part of the chapter deals with numerical integration, a tecnique used to computationally calculate the integral of physics quantities like acceleration and velocity to get the trajectory traveled by the vehicle.
The second part deals with quaternion, a number system useful to handle rotations in tridimensional space more efficient and numerically stable than rotation matrices.

\section{Numerical integration}

\section{Quaternions}
Quaternions are an extension of complex numbers usually represented in the form:
$$ a + xi + yj + wk $$
where $a, x, y, w$ are real numbers and $i, j, k$ are called \textit{quaternion units} \\
with the property ${ijk=-1}$
$a$ is also called \textit{scalar} or \textit{real} part of the quaternion, while $x,y,w$ is the tridimensional \textit{vector} or \textit{imaginary} part. \\
Quaternions have been formalized by the irish mathematician William Hamilton in 1843. \\
If the scalar part is equal zero then the quaternion is \textit{pure} \\