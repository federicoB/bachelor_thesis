%\let\textcircled=\pgftextcircled
\chapter{Introduction}
\label{chap:intro}

Vehicle dynamics reconstruction is a topic with much interest especially in insurance. A car crash can be recreated visually after the event happened. There are other topics with interest in vehicle dynamics reconstruction like self-driving vehicles or everything interested in knowing the vehicle position at some point in time.
This thesis will analyze techniques for reconstruction based on inertial and GNSS data.
Inertial data is created by accelerometers and gyroscopes, while GNSS by a GPS sensor.
All this sensor can be bundled in a box, as in our case, and fitted on a vehicle.

% insert photo of a sensor box

Challenges that aspiring solution may face are: sensor data gathering and transmission, correction of bad alignment of sensors, integration numerical error, precision reinforcement with multiple sensor fusion, representation of reconstructed trajectory.

% explain input data and the choose of blender