%\let\textcircled=\pgftextcircled
\chapter{Introduction}
\label{chap:intro}

Vehicle dynamics reconstruction is a topic with much interest especially in insurance makers. A car crash can be recreated visually after the event happened. But there are other topics with interest in vehicle dynamics reconstruction, like self-driving vehicles or everything interested in knowing the vehicle position at some point in time. \\
This thesis will analyze techniques for reconstruction based on inertial and GNSS data used in a related software project. \\
In the data collected and used for the project inertial data was created by accelerometers and gyroscopes, while GNSS by a GPS sensor. \\
All these sensors can be bundled in a box, as in our case, and fitted on a vehicle. \\

% insert photo of a sensor box

Challenges the aspiring solution may face are: sensor data gathering and transmissoin, correction of bad alignment of sensors, integration numerical error, precision reinforcement with multiple sensor fusion, representation of reconstructed trajectory.

% explain input data and the choose of blender