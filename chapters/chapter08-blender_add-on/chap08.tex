%\let\textcircled=\pgftextcircled
\chapter{Blender add-on}
\label{chap:blender_add_on}

Blender has a Python Application Programming Interface (API). Through the API, one can create a Python module and package it as Blender add-on. Add-ons are extension of Blender basic functionality, managed by a section of Blender settings and installable by file.
As alternative to add-on to interact programmatically with Blender is possible to create a Python script and load it into the built-in text editor (and execute it) or give it as parameter while launching Blender from the command line.
Add-ons provide versioning and authoring, remain activated between Blender files and don't need to be loaded each time.

\section{Blender API}
All the operations are made through \texttt{bpy} Python module.
\texttt{bpy} can be subdivided additionally into:
\begin{itemize}
\item \texttt{bpy.context}: access to current active scene and elements.
\item \texttt{bpy.data}: access to all objects in Blender document.
\item \texttt{bpy.types}: built-in Python classes, usually extended. %TODO for example...
\item \texttt{bpy.props}: abstraction of a variable properties, having type \texttt{bpy.types.*Property}; if registered in \texttt{bpy.props}, it can be modified by graphical interface.
\end{itemize}
From \texttt{bpy.types} other classes used in the projects are:
\begin{itemize}
\item Operator: essentialy a function to manipulate objects. Every action that the user can do in Blender is coded like an \texttt{Operator}: like moving, modelling and sculpting objects.
\item Panel: a section of the graphical interface.

% TODO other Blender api niceties like adding element or that you can look at Blender code.
\end{itemize}

\section{Blender add-on anatomy}
An add-on can be a single Python file or a zip containing an \texttt{\_\_init\_\_.py} file and other complementary files. \\
In both cases, the main Python file must have a \texttt{register()} and \texttt{unregister()} function implemented. \\
The \texttt{unregister()} takes care of calling \texttt{bpy.utils.unregister\_class()}, for each class that needs to be loaded into Blender. \\
When loading a class, Blender performs sanity checks, making sure all required properties and functions are found, that properties have the correct type and that functions have the right number of arguments. \texttt{register()} function of registered class are called recursively and, for example, \texttt{Panels }are drawn on user interface at this moment. \\
The \texttt{unregister()} have a totally opposite task compared to \texttt{register()}. \\
It should call \texttt{bpy.utils.unregister\_class()} on every register call. It's preferable to make the calls in the reverse order compared to \texttt{register()}. \\
Other objects defined in the Python add-on are:
\begin{itemize}
\item subclass of \texttt{Panel} for user-interface, a button to open a file system explorer dialog to select input dataset, an editable text field showing input dataset path and a button to launch dynamic reconstruction.
\item \texttt{StringProperty} for storing input dataset file path and to share it between operators.
\item subclasses of \texttt{Operator} for reaction to buttons clicks on user interface and calling related project functions. 
\end{itemize}
Finally, a Python dictionary called \texttt{bl\_info} is defined with add-on metadata as name, version and author. Those data are displayed in the list of add-ons in Blender settings.

\section{Installation of dependencies}
One particular challenge I faced was to have project library dependencies installed in blender internal interpreter. \\
There are three solution to this problem. Two solutions were discarded: 
\begin{itemize}
\item Removing the Blender Python folder and Blender will fallback to system interpreter. But API supported Python version can be different to the system one and doing this is dangerous.
\item Changing \textit{Scripts path} inside Blender setting, to look to an another directory with a specific structure. This directory must contains \texttt{add-ons, modules} and \texttt{startup} directories. Project dependencies should be installed in \texttt{modules} directory. There are various problems to this solution: it can be easily made non-working by a Blender update, removing the possibility of changing \textit{Scripts path}, also it can be against the final Blender user will, requiring all previous installed add-ons to be moved to new directory.
\end{itemize}
The solution I decided to follow and implement is to increase Blender interpreter capabilities by installing the \textbf{pip} package manager on it and then using \textit{pip} to install the dependencies listed in the \texttt{requirements.txt} file. The whole process can be labeled as \textit{bootstrapping}, as we start from a very small capability and we use it to have more at each step. \\
First step is to download a Python script which has \textit{pip} binary included inside. To do this it is very useful the \texttt{get-pip.py} script. \cite{get-pip}
Then, \texttt{get-pip.py} must be run by the Blender interpreter so it will be linked to it.  \\
In the Blender Python directory a pip executable will emerge so project dependencies can be installed with it and modules will be linked to. \\
Including the code to do this process in the add-on is crucial to have a single \texttt{zip} to distribute and increase final user experience. \\
This part has been a bit tricky due to differences between operating system; both for file system structure and operating system interfaces to execute programs. \\