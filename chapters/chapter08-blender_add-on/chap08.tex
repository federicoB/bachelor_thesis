%\let\textcircled=\pgftextcircled
\chapter{Blender add-on}
\label{chap:blender_add_on}

Blender has a python application programming interface (API). Through the API, one can create a python module as a Blender add-on. Add-ons are extension of Blender basic functionality, managed by a section of Blender settings and installable by file.
An alternative to add-on to interact programmatically with Blender is to create a python script and load it into the built-in text editor and execute it or by giving it as parameter launching Blender from the command line.
Add-ons provide versioning and authoring, remain activated through changing between Blender files and doesn't need to be loaded each time.

\section{Blender API}
All the operations are made through \texttt{bpy} python module.
\texttt{bpy} can be subdivided additionally mainly to:
\begin{itemize}
\item \texttt{bpy.context}: access to current active scene and elements.
\item \texttt{bpy.data}: access to all objects in Blender document.
\item \texttt{bpy.types}: built-in python classes, usually extended.
\item \texttt{bpy.props}: abstraction of a variable. properties of type \texttt{by.types.*Property} registered in \texttt{bpy.props} will be modifiable by graphical interface.
\end{itemize}
From \texttt{bpy.types} other classes used in the projects are:
\begin{itemize}
\item Operator: essentialy a function to manipulate objects.
\item Panel: a section of the graphical interface.
\end{itemize}

\section{Blender add-on anatomy}
Ad add-on can be a single python file or a zip containing an \texttt{\_\_init\_\_.py} file and other complementary files. \\
In both cases the main python file must have a \texttt{register()} and \texttt{unregister()} function implemented. \\
The \texttt{unregister()} takes care of calling \texttt{rbpy.utils.unregister\_class()} for each class that need to be loaded into Blender. \\
When loading a class, Blender performs sanity checks making sure all required properties and functions are found, that properties have the correct type and that functions have the right number of arguments. \texttt{register()} function of registered class are called recursively and, for example, \texttt{Panels }are drawn on user interface at this moment. \\
The \texttt{unregister()} have a totally opposite task compared to \texttt{register()}. \\
It should call \texttt{rbpy.utils.unregister\_class()} on every register call. It's preferable make the calls in the reverse order compared to \texttt{register()}. \\
Other objects defined in the python add-on are:
\begin{itemize}
\item subclass of \texttt{Panel} for user-interface, a button to open a file system explorer dialog to select input dataset, a editable text field showing input dataset path and a button to launch dynamic reconstruction.
\item \texttt{StringProperty} for storing input dataset file path and to sharing it between operators.
\item subclasses of \texttt{Operator} for reaction to pressing buttons on user interface and calling related project functions. 
\end{itemize}
Finally, a python dictionary called \texttt{bl\_info} is defined with add-on metadata as name, version and author. Those data is displayed in list of add-ons of Blender settings.

\section{Installation of dependencies}
One particular challenge I faced is to have project library dependencies installed in blender internal interpreter. \\
There are 3 solution to this problem. The 2 solutions i discarded are: 
\begin{itemize}
\item Removing the Blender python folder and Blender will fallback to system interpreter. But API supported python version can be different to system and doing this is dangerous.
\item Changing \textit{Scripts path} Blender setting, to look to an another directory with a specific structure. This directory must contains \texttt{add-ons, modules} and \texttt{startup} directories. Project dependencies should be installed in \texttt{modules} directory. There are various problems to this solution: it can be easily made non-working by a Blender update removing possibility of changing \textit{Scripts path}, it can be against Blender user will and previous installed add-ons must be moved to new directory.
\end{itemize}
The solution is decided to follow and implement is to increase Blender interpreter capabilities by installing the \textbf{pip} package manager on it and then using \textit{pip} to install the dependencies listed in the \texttt{requirements.txt} file. The whole process can be labeled as \textit{bootstrapping} as we start from a very small capabilities and we use it to have at each step more. \\
First step is to download a python script with \textit{pip} binary included inside. To do this is very useful the \texttt{get-pip.py} script. \cite{get-pip}
Then \texttt{get-pip.py} must be interpreted by the Blender interpreter so it will be linked to it.  \\
In the Blender python directory it will be created a pip executable so project dependencies can be installed with it and modules will be linked to. \\
Including the code to do this process in the add-on is crucial to have a single \texttt{zip} to distribute and increase final user experience. \\
This part has been a bit tricky for difference between operating system, both for difference in Blender installation and add-on paths and operating system interface to execute programs. \\