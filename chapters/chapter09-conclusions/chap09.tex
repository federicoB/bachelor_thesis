%\let\textcircled=\pgftextcircled
\chapter{Conclusions}
\label{chap:conclusions}

Thus thesis presented a software to reconstruct vehicle dynamics from inertial and GNSS data.
I developed an algorithm to integrate physical quantities with a technique suitable also for not equally time separated records. Both acceleration and angular velocity are integrated; angular position is composed using quaternion performance advantages.
Trajectory obtained from accelerometer and gyroscope is improved additionally by GNSS data, leading to a more precise and smooth path.
The project has been implemented in Python, mainly take advantage of Numpy and Pyquaternion libraries functionalities.
I created a standalone Blender add-on which auto-installs its dependencies and offers a graphical interface. Most of the code has been checked with automated testing.

\section{Future Development}
Several improvement can be done:
\begin{itemize}
\item angular position isn't corrected and with time tends to diverge due to numerical error, one can use GNSS data to correct horizontal orientation, $\theta_x$ and $\theta_y$ can be reset to zero if we detect the car is one plain surface. Online maps API can be used to have an approximation of road slope.
\item gyroscope offset drift can be additionally removed with using temperature sensor
\item vertical position can be detected using barometer
\item data from multiple boxes mounted on the same vehicle can be merged, once know the relative position
\item using approximation instead of interpolation for GNSS position determination, for reduce error when vehicle is stationary
\item change basis of second grade polynomial used in integration, for improve stability
\item parallelize execution, expecially in delta integration routine where it can be distributed on $n$ cores.