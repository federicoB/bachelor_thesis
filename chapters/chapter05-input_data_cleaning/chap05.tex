%\let\textcircled=\pgftextcircled
\chapter{Input data cleaning}
\label{chap:input_data_cleaning}

The input data from the sensor doesn't represent the real measure of vehicle physics quantities. Instruments have an error of measure, each one with different possible causes. This chapter deals with techniques used in the project to reduce that error, whether is caused by human accidentally or it's intrinsic of the sensor.

\section{Stationary time detection}
Most of the solution to remove errors explained later are based on assumption of the state of the vehicle. One the most used assumption is when the vehicle is stationary.
This can't be detected by a near zero acceleration along all axis because the vehicle could be moving with costant speed. So integrated speed should be used, but integrating acceleration vectors that need corrections can bring next code logic to make mistakes as well being a waste of time. 
One solution can be using non-directional speed from the input data, calculated from GNSS data by the sensor. But this speed is under effect of Kalman filters that react late to changes, still for trying to avoid measurement errors. I decided to derivate numerically the GNSS position to get a more precise speed and on top of that calculate stationary times.  

\section{Gyroscope drift}
Gyroscopes have a drift that is unavoidable. \cite{6727722}
There are various technique to remove that, the one used in this project is the rolling average. 
Given a series $s$ the centered rolling average with a window size of $w$ is definied as:
$$ v_i = \frac{1}{w} \sum_{j=0}^{\frac{w}{2}}v_j \sum_{j=\frac{w}{2}}^{w}v_j $$
where $v_i$ is the element of the series $s'$, that is $s$ with the centered rolling average applied.
The choice of a value for the windows size is important. A too small value can lead to a too high noise left, while a too high windows size value can lead to \textit{flattening} and reduction of measured value. For example with a too high value an acceleration start to be measured before it actually exist and the overall highest value is lower than in reality.

\section{Noise reduction}

\section{Correction of vertical alignment}

\section{Correction of horizontal alignment}