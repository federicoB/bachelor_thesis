\title{Reconstruction of vehicle dynamics from inertial and GNSS data}
\RequirePackage[l2tabu]{nag}		% Warns for incorrect (obsolete) LaTeX usage

\documentclass[a4paper,11pt,leqno,openbib,oldfontcommands]{memoir} %add 'draft' to turn draft option on (see below)
%
%
% Adding metadata:
\usepackage{datetime}
\usepackage{ifpdf}
\ifpdf
\pdfinfo{
   /Author (Federico Bertani)
   /Title (Reconstruction of vehicle dynamics from inertial and GNSS data)
   /Keywords (Vehicle dynamics; inertial data;GNSS data; blender; python; dynamics reconstruction)
   /CreationDate (D:\pdfdate)
}
\fi
% When draft option is on. 
\ifdraftdoc 
	\usepackage{draftwatermark}				%Sets watermarks up.
	\SetWatermarkScale{0.3}
	\SetWatermarkText{\bf Draft: \today}
\fi

% Better page layout for A4 paper, see memoir manual.
\settrimmedsize{297mm}{210mm}{*}
\setlength{\trimtop}{0pt} 
\setlength{\trimedge}{\stockwidth} 
\addtolength{\trimedge}{-\paperwidth} 
\settypeblocksize{634pt}{448.13pt}{*} 
\setulmargins{4cm}{*}{*} 
\setlrmargins{*}{*}{1.5} 
\setmarginnotes{17pt}{51pt}{\onelineskip} 
\setheadfoot{\onelineskip}{2\onelineskip} 
\setheaderspaces{*}{2\onelineskip}{*} 
\checkandfixthelayout
%
\frenchspacing
% Font with math support: New Century Schoolbook
\usepackage{fouriernc}
\usepackage[T1]{fontenc}
%
% UoB guidelines:
%
% Text should be in double or 1.5 line spacing, and font size should be
% chosen to ensure clarity and legibility for the main text and for any
% quotations and footnotes. Margins should allow for eventual hard binding.
%
% Note: This is automatically set by memoir class. Nevertheless \OnehalfSpacing 
% enables double spacing but leaves single spaced for captions for instance. 
\OnehalfSpacing 
%
% Sets numbering division level
\setsecnumdepth{subsection} 
\maxsecnumdepth{subsubsection}
%
% Chapter style (taken and slightly modified from Lars Madsen Memoir Chapter 
% Styles document
\usepackage{calc,soul,fourier}
\makeatletter 
\newlength\dlf@normtxtw 
\setlength\dlf@normtxtw{\textwidth} 
\newsavebox{\feline@chapter} 
\newcommand\feline@chapter@marker[1][4cm]{%
	\sbox\feline@chapter{% 
		\resizebox{!}{#1}{\fboxsep=1pt%
			\colorbox{gray}{\color{white}\thechapter}% 
		}}%
		\rotatebox{90}{% 
			\resizebox{%
				\heightof{\usebox{\feline@chapter}}+\depthof{\usebox{\feline@chapter}}}% 
			{!}{\scshape\so\@chapapp}}\quad%
		\raisebox{\depthof{\usebox{\feline@chapter}}}{\usebox{\feline@chapter}}%
} 
\newcommand\feline@chm[1][4cm]{%
	\sbox\feline@chapter{\feline@chapter@marker[#1]}% 
	\makebox[0pt][c]{% aka \rlap
		\makebox[1cm][r]{\usebox\feline@chapter}%
	}}
\makechapterstyle{daleifmodif}{
	\renewcommand\chapnamefont{\normalfont\Large\scshape\raggedleft\so} 
	\renewcommand\chaptitlefont{\normalfont\Large\bfseries\scshape} 
	\renewcommand\chapternamenum{} \renewcommand\printchaptername{} 
	\renewcommand\printchapternum{\null\hfill\feline@chm[2.5cm]\par} 
	\renewcommand\afterchapternum{\par\vskip\midchapskip} 
	\renewcommand\printchaptertitle[1]{\color{gray}\chaptitlefont\raggedleft ##1\par}
} 
\makeatother 
\chapterstyle{daleifmodif}
%
% UoB guidelines:
%
% The pages should be numbered consecutively at the bottom centre of the
% page.
\makepagestyle{myvf} 
\makeoddfoot{myvf}{}{\thepage}{} 
\makeevenfoot{myvf}{}{\thepage}{} 
\makeheadrule{myvf}{\textwidth}{\normalrulethickness} 
\makeevenhead{myvf}{\small\textsc{\leftmark}}{}{} 
\makeoddhead{myvf}{}{}{\small\textsc{\rightmark}}
\pagestyle{myvf}
%
% Oscar's command (it works):
% Fills blank pages until next odd-numbered page. Used to emulate single-sided
% frontmatter. This will work for title, abstract. Though the
% contents sections will each start on an odd-numbered page they will
% spill over onto the even-numbered pages if extending beyond one page
% (hopefully, this is ok).
\newcommand{\clearemptydoublepage}{\newpage{\thispagestyle{empty}\cleardoublepage}}
%
%
% Creates indexes for Table of Contents, List of Figures, List of Tables and Index
\makeindex
% \printglossaries below creates a list of abbreviations. \gls and related
% commands are then used throughout the text, so that latex can automatically
% keep track of which abbreviations have already been defined in the text.
%
% The import command enables each chapter tex file to use relative paths when
% accessing supplementary files. For example, to include
% chapters/brewing/images/figure1.png from chapters/brewing/brewing.tex we can
% use
% \includegraphics{images/figure1}
% instead of
% \includegraphics{chapters/brewing/images/figure1}
\usepackage{import}

% Add other packages needed for chapters here. For example:

\usepackage{amsfonts} 		{}			%Calls Amer. Math. Soc. (AMS) fonts
\usepackage[centertags]{amsmath}			%Writes maths centred down
\usepackage{layouts}					%Layout diagrams
\usepackage{graphicx}					%Calls figure environment
\usepackage{wrapfig}
\usepackage{ragged2e}
\usepackage{longtable}			%Long tab environments including rotation. 
\usepackage[utf8]{inputenc}			%Needed to encode non-english characters 
\usepackage{colortbl}					%Makes coloured tables
\usepackage{float}						%Helps to place figures, tables, etc. 
\floatstyle{boxed} 
\restylefloat{figure}
\usepackage{verbatim}					%Permits pre-formated text insertion
\usepackage[style=numeric,maxnames=50]{biblatex}
\usepackage{url}{{}}						%Supports url commands
\usepackage[english]{babel}		%For languages characters and hyphenation
\usepackage{color}                                             %Creates coloured text and background
\usepackage[colorlinks=true,
                    allcolors=black,
                    bookmarks]{hyperref}              %Creates hyperlinks in cross references
\usepackage{memhfixc}					%Must be used on memoir document class after hyperref
\usepackage{enumerate}					%For enumeration counter
\usepackage{footnote}					%For footnotes
\usepackage{microtype}					%Makes pdf look better.
\usepackage{alltt}						%LaTeX commands are not disabled in verbatim-like environment
\usepackage{listings}				% for source code
	
%							
%Reduce widows  (the last line of a paragraph at the start of a page) and orphans 
% (the first line of paragraph at the end of a page)
\widowpenalty=1000
\clubpenalty=1000


\bibliography{federico_bertani_tesi}

%
%
\begin{document}
% UoB guidlines:
%
% Preliminary pages
% 
% The five preliminary pages must be the Title Page, Abstract, Dedication
% and Acknowledgements, and Table of Contents.
% These should be single-sided.
% 
% Table of contents, list of tables and illustrative material
% 
% The table of contents must list, with page numbers, all chapters,
 % sections and subsections, the list of references, bibliography, list of
% abbreviations and appendices. The list of tables and illustrations
% should follow the table of contents, listing with page numbers the
% tables, photographs, diagrams, etc., in the order in which they appear
% in the text.
% 
\frontmatter
\pagenumbering{roman}
%
%
% File: Title.tex
% Author: V?ctor Bre?a-Medina
% Description: Contains the title page
%
% UoB guidelines:
% 
% At the top of the title page, within the margins, the dissertation should give the title and, if 
% necessary, sub-title and volume number. If the dissertation is in a language other than English, the 
% title must be given in that language and in English. The full name of the author should be in the 
% centre of the page. At the bottom centre should be the words ?A dissertation submitted to the 
% University of Bristol in accordance with the requirements for award of the degree of ? in the 
% Faculty of ...?, with the name of the school and month and year of submission. The word count of 
% the dissertation (text only) should be entered at the bottom right-hand side of the page.
%
%
\begin{titlingpage}
\begin{SingleSpace}
\calccentering{\unitlength} 
\begin{adjustwidth*}{\unitlength}{-\unitlength}
\vspace*{13mm}
\begin{center}
\rule[0.5ex]{\linewidth}{2pt}\vspace*{-\baselineskip}\vspace*{3.2pt}
\rule[0.5ex]{\linewidth}{1pt}\\[\baselineskip]
{\HUGE University of Bristol Thesis Template }\\[4mm]
{\Large \textit{Subtitle}}\\
\rule[0.5ex]{\linewidth}{1pt}\vspace*{-\baselineskip}\vspace{3.2pt}
\rule[0.5ex]{\linewidth}{2pt}\\
\vspace{6.5mm}
{\large By}\\
\vspace{6.5mm}
{\large\textsc{Author's name}}\\
\vspace{11mm}
\includegraphics[scale=0.6]{logos/bristolcrest_colour}\\
\vspace{6mm}
{\large Department of Engineering Mathematics\\
\textsc{University of Bristol}}\\
\vspace{11mm}
\begin{minipage}{10cm}
A dissertation submitted to the University of Bristol in accordance with the requirements of the degree of \textsc{Doctor of Philosophy} in the Faculty of Engineering.
\end{minipage}\\
\vspace{9mm}
{\large\textsc{April 2013}}
\vspace{12mm}
\end{center}
\begin{flushright}
{\small Word count: ten thousand and four}
\end{flushright}
\end{adjustwidth*}
\end{SingleSpace}
\end{titlingpage}
\clearemptydoublepage
%
% Description: Contains the text for thesis abstract
%
% Abstract guidelines:
%
% Each copy must include an abstract or summary of the dissertation in not
% more than 300 words, on one side of A4, which should be single-spaced in a
% font size in the range 10 to 12. If the dissertation is in a language other
% than English, an abstract in that language and an abstract in English must
% be included.

\chapter*{Abstract}
\begin{SingleSpace}

The increasingly massive collection of data from various types of sensors installed on vehicles allows the study and reconstruction of their dynamics in real time, as well as their archiving for subsequent analysis.
This Thesis describes the design of a numerical algorithm and its implementation in a program that uses data from inertial and geo-positioning sensors, with applications in industrial contexts and automotive research. The result was made usable through the development of a Python add-on for the Blender graphics program, able to generate a three-dimensional view of the dynamics that can be used by experts and others.
Throughout the Thesis, particular attention was paid to the complex nature of the data processed, introducing adequate systems for filtering, interpolation, integration and analysis, aimed at reducing errors and simultaneously optimizing performances.

\vspace{20mm}

La raccolta sempre più massiccia di dati provenienti da sensori di varia natura installati sui veicoli in circolazione permette lo studio e la ricostruzione della loro dinamica in tempo reale, nonché la loro archiviazione per analisi a posteriori. 
In questa Tesi si descrive la progettazione di un algoritmo numerico e la sua implementazione in un programma che utilizza dati provenienti da sensori inerziali e di geo-posizionamento con applicazioni a contesti industriali e di ricerca automobilistica. Il risultato è stato reso fruibile tramite lo sviluppo di un add-on Python per il programma di grafica Blender, in grado di generare una visualizzazione tridimensionale della dinamica fruibile da esperti e non. 
Durante tutto il lavoro di Tesi, particolare attenzione è stata prestata alla complessa natura dei dati trattati, introducendo adeguati sistemi di filtraggio, interpolazione, integrazione ed analisi, volti alla riduzione degli errori e alla contemporanea ottimizzazione delle prestazioni.

\end{SingleSpace}
\clearpage
%
\epigraph{\textit{I'm personally convinced that computer science has a lot in common with physics. Both are about how the world works at a rather fundamental level. The difference, of course, is that while in physics you're supposed to figure out how the world is made up, in computer science you create the world. Within the confines of the computer, you're the creator. You get to ultimately control everything that happens. If you're good enough, you can be God. On a small scale.}}{Linus Torvalds}
%
\clearpage
%
\renewcommand{\contentsname}{Table of Contents}
\maxtocdepth{subsection}
\tableofcontents
\vfill
\justify
\textbf{Please visit \url{https://github.com/federicoB/bachelor_thesis} to find an updated version of this document}
\clearpage


%
% The bulk of the document is delegated to these chapter files in
% subdirectories.
\mainmatter
%
\import{chapters/chapter01-introduction/}{chap01.tex}

\import{chapters/chapter02-mathematical_background/}{chap02.tex}

\import{chapters/chapter03_language_and_libraries_choices/}{chap03.tex}

\import{chapters/chapter04-input_data_cleaning/}{chap04.tex}

\import{chapters/chapter05-rotations_with_quaternions/}{chap05.tex}

\import{chapters/chapter06-trajectory_integration/}{chap06.tex}

\import{chapters/chapter07-blender_add-on/}{chap07.tex}

\import{chapters/chapter08-software_engineering_considerations/}{chap08.tex}

\import{chapters/chapter09-conclusions/}{chap09.tex}

% description: Contains the text for thesis dedication

\chapter*{Ringraziamenti}
\begin{SingleSpace}
Vorrei ringraziare tutto il gruppo di Fisica dei Sistemi Complessi, in specifico Stefano Sinigardi, Alessandro Fabbri, Nico Curti e Raffaele Pepe per il supporto e per il bel ambiente lavorativo in cui sono stato. \\
Vorrei ringraziare la mia famiglia per l'aiuto che mi hanno dato e perché mi hanno permesso di studiare seguendo le mie passioni.
\end{SingleSpace}
\clearpage
\clearpage
%
% Apparently the guidelines don't say anything about citations or
% bibliography styles so I guess we can use anything.
\backmatter

\refstepcounter{chapter}
\renewcommand{\bibname}{References}
\printbibliography
%
% Add index
%\printindex
%   
\end{document}